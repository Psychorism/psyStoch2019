\documentclass[12pt]{article} 

\usepackage{geometry}
\geometry{a4paper} 

\usepackage{graphicx} 
\usepackage{enumitem}
\usepackage{booktabs}

\usepackage{float} 
\usepackage{wrapfig} 

\usepackage{amsmath}
\usepackage{amsfonts}
\usepackage{amssymb}
\usepackage{dsfont}

\usepackage{xcolor}
\usepackage{listings}
\usepackage{caption}
\DeclareCaptionFont{white}{\color{white}}
\DeclareCaptionFormat{listing}{%
  \parbox{\textwidth}{\colorbox{gray}{\parbox{\textwidth}{#1#2#3}}\vskip-2pt}}
\captionsetup[lstlisting]{format=listing,labelfont=white,textfont=white}
\lstset{frame=lrb,xleftmargin=\fboxsep,xrightmargin=-\fboxsep}

\linespread{1.2} 
\setlength{\parskip}{\baselineskip} % vertical spaces
\setlength\parindent{0pt} % remove all indentation from paragraphs


\usepackage{ntheorem}
\usepackage{mdframed}

\theoremstyle{nonumberbreak}
\theoremheaderfont{\bfseries}
\newmdtheoremenv[%
linecolor=gray,leftmargin=10,%
rightmargin=10,
backgroundcolor=gray!20,%
innertopmargin=0pt,%
ntheorem]{theorem}{}




\begin{document}

\title{\textbf{The Queueing Theory}}
\author{Hyunwoo Gu}
\date{}

\maketitle


%----------------------------------------------------------------------------------------
%   Section 1
%----------------------------------------------------------------------------------------
\section{Definitions}

The \textbf{first} character describes the \textbf{arrival process} to the queue.

\begin{itemize}
	\item \textbf{M}emorylessness: Poisson arrival process\item \textbf{D}eterministic: interarrival interval is nonrandom
	\item \textbf{G}eneral: general interarrival distribution
\end{itemize}

The \textbf{second} character describes the \textbf{service process}, with $M$: exponentially distributed service times. 

The \textbf{third} character describes the \textbf{number of servers}.  

Sometimes a fourth character is added which gives the \textbf{number of customers} that can be saved in the queue plus service facility. 

It is often assumed that the service times are IID, independent of the arrival epochs, and independent of which server is used. 


\textbf{M/M/1 queue}(Gallager). A queueing system with a Poisson arrival system with rate $\lambda$ and a single server that serves arriving customers in order with a service time distribution $F(y) = 1 - exp(-\mu y)$. 

The service times are IID and independent of the interarrival intervals. During any period when the server is busy, customers leave the system according to a Poisson process



\textbf{M/G/$\infty$ queue}(Gallager). A queueing system with Poisson arrivals, a general service distribution, and an infinite number of servers. From the infinite number of servers, no arriving customers are ever queued. The service time $Y_i$ of customer $i$ is IID over $i$ with CDF $G(y)$. 


%----------------------------------------------------------------------------------------
%   Section 2
%----------------------------------------------------------------------------------------


\section{Examples}

\subsection{Example 1}

Let us consider the following queueing model, where the transition matrix is given as

$$
P_{ij} = \begin{bmatrix}
\end{bmatrix}
$$

This decomposition, in conjunc­tion with the Markov property, yields the following expression for the average recurrence time:


\textbf{Case 1}. 


Referring again to the queueing process., we identify the $b$'s and the $a$'s by $a_k = bk _ 1$ and let $Z_{ij} =$ length of time (number of transitions) re­quired., starting in state $i$, to reach state $j < i$ for the first time. 

is precisely the random variable $Z$ examined above whose generating function $U(s)$ was determined. Since


\subsection{Example 2}

The state of the process is the length of the waiting line where, in each unit of time, one person arrives and $k$ persons are served with probability $a_k > 0$, $k=0,1,\cdots$. 


if there are at least transition probability matrix may be easily evaluated as



Therefore, if $L kak < 1$ the system admits no bounded solution and therefore, in particular, no stationary distribution exists so that the process is either null recurrent or transient. 

We now prove that the system of equation


To sum up, if

\begin{itemize}
	\item $\sum k a_k < 1$, the process is \textbf{transient}
	\item $\sum k a_k = 1$, the process is \textbf{null recurrent}
	\item $\sum k a_k > 1$, the process is \textbf{positive recurrent}
\end{itemize}

$$
\begin{aligned}
\sum k a_k <1 &= 1, \\[8pt]
\end{aligned}
$$





\end{document}