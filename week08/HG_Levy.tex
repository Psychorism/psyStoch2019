\documentclass[12pt]{article} 

\usepackage{geometry}
\geometry{a4paper} 

\usepackage{graphicx} 
\usepackage{enumitem}
\usepackage{booktabs}

\usepackage{float} 
\usepackage{wrapfig} 

\usepackage{amsmath}
\usepackage{amsfonts}
\usepackage{amssymb}
\usepackage{dsfont}

\usepackage{xcolor}
\usepackage{listings}
\usepackage{caption}
\DeclareCaptionFont{white}{\color{white}}
\DeclareCaptionFormat{listing}{%
  \parbox{\textwidth}{\colorbox{gray}{\parbox{\textwidth}{#1#2#3}}\vskip-2pt}}
\captionsetup[lstlisting]{format=listing,labelfont=white,textfont=white}
\lstset{frame=lrb,xleftmargin=\fboxsep,xrightmargin=-\fboxsep}

\linespread{1.2} 
\setlength{\parskip}{\baselineskip} % vertical spaces
\setlength\parindent{0pt} % remove all indentation from paragraphs


\usepackage{ntheorem}
\usepackage{mdframed}

\theoremstyle{nonumberbreak}
\theoremheaderfont{\bfseries}
\newmdtheoremenv[%
linecolor=gray,leftmargin=10,%
rightmargin=10,
backgroundcolor=gray!20,%
innertopmargin=0pt,%
ntheorem]{theorem}{}




\begin{document}

\title{\textbf{Levy processes}}
\author{Hyunwoo Gu}
\date{}

\maketitle


%----------------------------------------------------------------------------------------
%   Section 1
%----------------------------------------------------------------------------------------
\section{Introduction to the theory of Levy processes}

Here we will be able to: 

\begin{itemize}
	\item understand the main properties of Levy processes
	\item construct ta Levy process from an infinitely 
\end{itemize}


\subsection{Different types of stochastic integrals}

\subsection{Examples of Levy processes}

The class of \textbf{infinitely divisible distributions}. 




\begin{theorem}
\textbf{Definition}. An RV $\xi$ is a \textbf{infinitely divisible distribution}, if 

$$
\xi \overset{d}{\equiv} Y_1 \oplus \cdots \oplus Y_n
$$

Note that 

$$
\phi_\xi(u) = 
$$


\end{theorem}



It is not true that the distribution is stable iff it is infinitely divisible. 



\begin{theorem}
\textbf{Definition}. An RV $\xi$ is a \textbf{infinitely divisible distribution}, if 

$\exists \psi: \mathbb{R} \to \mathbb{C}$ such that

$$
\phi_{L_t} (u) = \mathbb{E} \left[ e^{iuL_t} \right] = e^{t \psi(u)}
$$

\end{theorem}


\textbf{Example 1}. 




\textbf{Example 12}. 


\subsection{Characteristic Exponent}





\subsection{Properties of Levy Characteristic Exponent}

\textbf{Levy measure}. 


\subsection{Levy-Khintchine triplet}


$$
\phi_{X_t}(u) = exp \left[ t (iu\mu - \sigma^2 u^2 /2) + \int_\mathbb{R} \left( e^{iux} + 1 - iux \mathbf{1}_{|x| < 1} \right) \right]
$$

\textbf{Example 1}. For $X_t$ : of bounded variation,

$$
\sum_{k=1}^n | X_{t_k} - X_{t_{k-1}} |  \overset{max |t_i - t_{i-1}| \to 0}{\to} 
$$


$\sigma=0$

Note that the Brownian motion is not a process of bounded variation.




\subsection{Modeling of jump-type}

How to estimate that the Levy measure of a Levy process from the following data? 

$$
X_t: X_\Delta
$$


Levy process can be used to model the properties of jobs. 


Let $X_t$ be of bounded variation, and 

$$
\phi_{X_\Delta} (u) := exp \left\{  \Delta \left( iu\mu + \int_\mathbb{R} (e^{iux} - 1) S(x) dx \right)  \right\}
$$

Note that 

$$
\phi_{X_\Delta} (u) := exp \left\{  \Delta \left( iu\mu + \int_\mathbb{R} (e^{iux} - 1) S(x) dx \right)  \right\}
$$



\subsection*{Quizzes}


\textbf{(Quiz 1)}. Let $X_t := cos(wt + \theta)$ be a stochastic process and $\theta \sim Unif(0,2\pi)$, with $w=\pi/10$. Classify this process.

\textbf{(Answer)} \textbf{Ergodic} and \textbf{weak stationary}. 


\textbf{(Quiz 2)}. Let $X_t := \epsilon_t + \xi cos(\pi t/12)$, $t=1,2,\cdots$, where $\xi, \epsilon_1, \epsilon_2, \cdots$ are IID standard normal random variables.

\textbf{(Answer)} Not \textbf{weak stationry}, but \textbf{ergodic}.


\textbf{(Quiz 3)}. Assume that for a process $X_t$ it is known that $\mathbb{E} (X_t) = \alpha + \beta t$, $cov(X_t, X_{t+h} = e^{-h \lambda}$ for all $h \ge 0, t >0$, and some constants $\lambda >0, \alpha, \beta$. Classify the process $Y_t := X_{t+1} - X_t$. 

\textbf{(Answer)} $Y_t$ is weakly stationary and ergodic.


\textbf{(Quiz 4)}. Let $X_t := \sigma W_t + ct$, where $W_t$ is Brownian motion, $\sigma, c >0$. Choose the correct statements about this process. 

\textbf{(Answer)} $X_t$ has \textbf{continuous trajectories}. 


\textbf{(Quiz 5)}. Let $X_t$ have an autocovariance function $\gamma(r) := e^{-\alpha |r|}$. Is $Y_t := X_t + w$ an ergodic process?


\textbf{(Answer)} Yes, if $w$ is a constant. 



\end{document}