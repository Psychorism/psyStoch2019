\documentclass[12pt]{article} 

\usepackage{geometry}
\geometry{a4paper} 

\usepackage{graphicx} 
\usepackage{enumitem}
\usepackage{booktabs}

\usepackage{float} 
\usepackage{wrapfig} 

\usepackage{amsmath}
\usepackage{amsfonts}
\usepackage{amssymb}
\usepackage{dsfont}

\usepackage{xcolor}
\usepackage{listings}
\usepackage{caption}
\DeclareCaptionFont{white}{\color{white}}
\DeclareCaptionFormat{listing}{%
  \parbox{\textwidth}{\colorbox{gray}{\parbox{\textwidth}{#1#2#3}}\vskip-2pt}}
\captionsetup[lstlisting]{format=listing,labelfont=white,textfont=white}
\lstset{frame=lrb,xleftmargin=\fboxsep,xrightmargin=-\fboxsep}

\linespread{1.2} 
\setlength{\parskip}{\baselineskip} % vertical spaces
\setlength\parindent{0pt} % remove all indentation from paragraphs


\usepackage{ntheorem}
\usepackage{mdframed}

\theoremstyle{nonumberbreak}
\theoremheaderfont{\bfseries}
\newmdtheoremenv[%
linecolor=gray,leftmargin=10,%
rightmargin=10,
backgroundcolor=gray!20,%
innertopmargin=0pt,%
ntheorem]{theorem}{}




\begin{document}

\title{\textbf{Supplementary: \\ Basics of Stochastic Processes}}
\author{Hyunwoo Gu}
\date{}

\maketitle


%----------------------------------------------------------------------------------------
%   Section 0
%----------------------------------------------------------------------------------------

\section{Basic measure theory}

\textbf{Lebesgue integration}

We say $f$ is \textbf{Lebesgue integrable} if 

$$
\int |f| d\mu < \infty
$$ 

We can \textbf{construct} the Lebesgue integral by using \textbf{simple functions}, which approximate a measurable funcion. They are basically a finite linear combination of indicator functions,

$$
\sum_k a_k \mathbf{1}_{S_k}
$$

where $S_k$ : measurable. Under $0 \times \infty =0$, we can define

$$
\int \left( \sum_k a_k \mathbf{1}_{S_k} \right) d\mu= \sum_k a_k \int \mathbf{1}_{S_k} d\mu = \sum_k a_k \mu(S_k)
$$

If $B$: a measurable subset of $\Omega$ and $s$ is a measurable simple function, one defines

$$
\int_B s d\mu = \int 1_B s d\mu = \sum_k a_k
 \mu(S_k \cap B)$$

Let $f$ be a nonnegative measurable function on $\Omega$, a subset of the extended real number line. We \textbf{define}

$$
\int_\Omega f d\mu = \mathrm{sup} \left\{ \int_\Omega s d\mu : 0 \le s \le f \right\}
$$

where $s$ is constrained to be simple. Alternatively, let $s_n(x)$ be the simple function whose value is $k/2^n$ whenever $k/2^n \le f(x) < (k+1)/2^n$ for $k$ a non-negative integer less than, say $4^n$. 

$$
\int f d\mu = \mathrm{lim}_{n\to\infty} \int s_n d\mu
$$

Note that we can define, for general $f$,

$$
\int f d\mu = \int f^+ d\mu - \int f^- d\mu
$$


\pagebreak
\textbf{Lebesgue-Stieltjes integration}

A generaliation of Riemann-Stieltjes integration and Lebesgue integration, which is just the ordinary Lebesgue integral with respect to a measure known as the \textbf{Lebesgue–Stieltjes measure}. 

The Lebesgue-Stieltjes integration

$$
\int_a^b f(x) d G(x)
$$

is defined when $f: [a,b] \to \mathbb{R}$ is \textbf{Borel-measurable} and \textbf{bounded}, and $G : [a,b] \to \mathbb{R}$ is of bounded variation



Assume $G(\cdot)$ is a right continuous step function having jumps at $x_1,x_2,\cdots$. Then

$$
\int_a^b f(x) dG(x) = \sum_{a < x_j \le b} f(x_j) \{ G(x_j) - G(x_j^-) \} = \sum_{a < x_j \le b} f(x_j) \Delta G(x_j)
$$


Let $F : \mathbb{R} \to \mathbb{R}$ be a distribution function. Then \textbf{there uniquely exists} a measure $\mu_F$ on $(\mathbb{R}, \mathcal{B} (\mathbb{R})$ s.t. 

$$
\mu_F \left((a,b] \right) = F(b) - F(a), \ \ \forall a < b \in \mathbb{R}
$$

Thus, 

$$
\int x dF(x) = \int x \mu_F (dx) = \int x d \mu_F(x)
$$

where the last equality is a \textbf{Lebesgue integral}. Note that $P_X\left( (a,b] \right) = \mu_F \left( (a,b] \right)$ for $X \sim F$. Thus

$$
P_X = \mu_F
$$

Thus we have

$$
\mathbb{E}(X) = \int_\Omega X dP_X = \int_\mathbb{R} x P_X (dx) = \int_\mathbb{R} x \mu_F (dx) = \int_\mathbb{R} x dF(x)
$$


Note that 

$$
\mathbb{E}[X^k] = M(0)
$$

where $M(t) = \sum_{l=0}^\infty \frac{\mathbb{E} (X^l)}{l!} t^k, \forall t \in (-\epsilon, \epsilon)$, $\exists \epsilon >0$.



\textbf{Radon-Nikodym theorem}

If $\nu << \mu$, then there is a \textbf{measurable function} $f: X \to [0, \infty)$ such that for any measurable set $A \subset X$, 

$$
\nu(A) = \int_A f d\mu
$$

For example, KL divergence from $\mu to \nu$ is defined

$$
D_{KL} (\mu \Vert \nu) = \int_\mathcal{X} \mathrm{log} \left( \frac{d\mu}{d\nu} d\mu \right)
$$


For another example, let $\Omega = [0,1]$, and $\mathbb{P}$ is uniform on $\Omega$. Define $X:[0,1]\to\mathbb{R}$ by $X(w) = -\log w$. Under $\mathbb{P}$ the rv $X$ has an $\mathrm{Exp}(1)$ distribution. Alternative measure $Q$ assigns $Q[a,b] = b^2 - a^2$. Under $Q$ the rv $X$ has an $\mathrm{Exp}(2)$ distribution: 

$$
Q\{X\leq x\} = Q\{w:-\log w\leq x\} = Q[e^{-x}, 1] = 1^2 - (e^{-x})^2 = 1 - e^{-2x}
$$

The Radon-Nikodym derivative of $Q$ wrt $\mathbb{R}$ is $\frac{dQ}{d\mathbb{P}}(\omega) = 2\omega$.

Generally, the Radon Nikodym derivative $f$ of $Q$ with respect to $P$ is defined by the equation 

$$
Q(E)=\int_E fdP
$$ 

for every measurable set $E$, if $Q << P$ In order find what this $f$ is it is enough consider the sets $E=[0,x]$ where $0\leq x \leq 1$. 

Thus we have to find $f$ such that $1-e^{-2x} =Q(X\leq x)=\int_{[0,x]} f(t)dt$. [ Note that $P$ is just the uniform measure (i.e. the Lebesgue measure on $[0,1]$ so $\int_E fdP=\int_E f(y)dy)$. 

To find $f$ from the equation $1-e^{-2x} =\int_0^{x} f(y)dy$ simply differentiate both sides with respect to $x$. Hence $f(x)=2e^{-2x}$.


Note that for the \textbf{cumulative intensity process}, we have

$$
d \Lambda(t) = \lambda(t) dt
$$

However, generally we cannot define

$$
\frac{dN(t)}{dt}
$$

since $N(t)$ is not absolutely continuous wrt $t$.

\end{document}